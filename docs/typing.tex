\documentclass{article}

% \prooftree and related commands
\usepackage{ebproof}
% most symbols
\usepackage{amsmath}
\usepackage{amssymb}
\usepackage{amsthm}
% \RHD
\usepackage[nointegrals]{wasysym}
\usepackage{xcolor}

\newcommand{\ignore}[1]{}
\newcommand{\todo}[1]{\colorbox{red}{#1}}
\newcommand{\consider}[1]{\colorbox{red}{!!!} #1 \colorbox{red}{!!!}}

\newtheorem{theorem}{Theorem}
\newtheorem{lemma}{Lemma}

\newcommand{\define}{::=}
\newcommand{\G}{\Gamma}
\newcommand{\St}{\mathcal{S}}
\newcommand{\unit}{\mathbf{1}}
\newcommand{\tp}{t}
\newcommand{\tarr}{t_{\text{arr}}}
\newcommand{\targ}{t_{\text{arg}}}
\newcommand{\tret}{t_{\text{ret}}}
\newcommand{\uv}{u}
\newcommand{\eu}{eu}
\newcommand{\e}{e}
\newcommand{\x}{x}
\newcommand{\var}{v}

\newcommand{\nil}{\cdot}
\newcommand{\spc}{\qquad}
\newcommand{\eq}{==}
\renewcommand{\implies}{\Rightarrow}
\newcommand{\fresh}{\text{Fresh}\ }

\newcommand{\set}[2]{{#1} = {#2}}
\newcommand{\withtp}[2]{{#1} \,:\, {#2}}
\newcommand{\app}[2]{{#1}\ {#2}}
\newcommand{\lam}[2]{\lambda {#1} \,.\, {#2}}
\newcommand{\lamtp}[3]{\lambda \left( {\withtp {#1} {#2}} \right) \,.\, {#3}}
\newcommand{\arr}[2]{{#1} \to {#2}}

\newcommand{\hastp}[3]{#1 \vdash {\withtp {#2} {#3}}}
\newcommand{\algtp}[6]{#1, #2 \vdash #3 \implies {\withtp {#4} {#5}} \dashv #6}
\newcommand{\equals}[4]{{#1} \vdash {#2} \eq {#3} \dashv {#4}}
\newcommand{\apply}[1]{\left[{#1}\right]}

\newcommand{\deduct}[3][]
{
  \begin{prooftree}
    \hypo{#2}
    \infer1[\text{#1}]{#3}
  \end{prooftree}
}

\begin{document}

\section{Grammar}

\subsection{Untyped Expressions}
\[
  \begin{array}{rcll}
    \eu & \define & \unit & \text{Unit} \\
        & \mid    & \var  & \text{Var}  \\
        & \mid    & \lam \x \eu & \text{Lambda} \\
        & \mid    & \app {\eu_1} {\eu_2} & \text{App}
  \end{array}
\]

\subsection{Typed Expressions}
\[
  \begin{array}{rcll}
    \e  & \define & \unit & \text{Unit} \\
        & \mid    & \var  & \text{Var}  \\
        & \mid    & \lamtp \x \tp \e & \text{Lambda} \\
        & \mid    & \app {\e_1} {\e_2} & \text{App}
  \end{array}
\]

\subsection{Types}
\[
  \begin{array}{rcll}
    \tp & \define & \unit & \text{Unit} \\
        & \mid    & \arr {\tp_1} {\tp_2}    & \text{Arrow}  \\
        & \mid    & \uv & \text{UVar} \\
  \end{array}
\]

\subsection{Typing Evironments}
\[
  \begin{array}{rcll}
    \G  & \define & \nil & \text{Empty} \\
        & \mid    & \G , {\withtp \var \tp} & \text{Var with Type}  \\
  \end{array}
\]

\subsection{Type Stores}
\[
  \begin{array}{rcll}
    \St  & \define & \nil & \text{Empty} \\
         & \mid    & \St , {\set \uv \tp} & \text{UVar with Type}  \\
  \end{array}
\]

\section{Typing}

\subsection{Declarative Typing}

Judgements of the form:

\[ \hastp \G \e \tp \]

\[
\deduct
  {}
  {\hastp \G \unit \unit}
  {\text{(DeclUnit)}}
\spc
\deduct
  {\G(\var) = \tp}
  {\hastp \G \var \tp}
  {\text{(DeclVar)}}
\]

\[
  \deduct
  {\hastp {\G, {\withtp \x {\tp_1}}} \e {\tp_2}}
  {\hastp \G {\lamtp {\x} {\tp_1} {\e}} {\arr {\tp_1} {\tp_2}}}
  {\text{(DeclLam)}}
\]

\[
  \deduct
  {\hastp \G {\e_1} {\arr {\tp_1} {\tp_2}} \spc 
   \hastp \G {\e_2} {\tp_1}}
  {\hastp \G {\app {\e_1} {\e_2}} {\tp_2}}
  {\text{(DeclApp)}}
\]

\subsection{Type stores}

We define the relation \(\St_1 \subseteq \St_2\), which states informally that
\(\St_1\) is a prefix of \(\St_2\). The Nil rule is not strictly needed.

\[
  \deduct
  {}
  {\nil \subseteq \St}
  {\text{(Nil)}}
  \spc
  \deduct
  {}
  {\St \subseteq \St}
  {\text{(Refl)}}
  \spc
  \deduct
  {\St_1 \subseteq \St_2}
  {\St_1 \subseteq \St_2, \uv = \tp}
  {\text{(Extension)}}
\]

It follows from the definition that the transitive propery holds for
\(\subseteq\).

We define applying a type store \(\St\) over type contexts, types,
and typed expressions.

For contexts \(\G\),
\[
  \begin{array}{rcll}
    \St(\nil) & \define & \nil & \text{Empty} \\
    \St(\G, {\withtp \var \tp}) & \define & \St(\G), {\withtp \var {\St(\tp)}} &
    \text{Var with Type}
  \end{array}
\]

For types \(\tp\),
\[
  \begin{array}{rcll}
    \St(\unit) & \define & \unit & \text{Unit} \\
    \St(\arr {\tp_1} {\tp_2}) & \define   & \arr {\St(\tp_1)} {\St(\tp_2)} & \text{Arrow}  \\
    \left( \St, {\uv_1} = \tp \right)({\uv_2}) & \define & \begin{cases}
    \St({\uv_2}) & {\uv_1} \ne {\uv_2} \\ \St(\tp) & {\uv_1} = {\uv_2}
    \end{cases} & \text{UVar} \\
    \nil(\tp) & \define & \tp & \text{Empty} \\
  \end{array}
\]

Note: though it is assumed that no UVar occurs twice in \(\St\), in the case of
multiple UVars, the first appearing one is used (from left to right, i.e. first-defined
UVar).

For typed expressions \(\e\),
\[
  \begin{array}{rcll}
    \St(\unit)  & \define & \unit & \text{Unit} \\
    \St(\var) & \define   & \var  & \text{Var}  \\
    \St(\lamtp \x \tp \e) & \define    & \lamtp \x {\St(\tp)} {\St(\e)} & \text{Lambda} \\
    \St(\app {\e_1} {\e_2}) & \define    & \app {\St(\e_1)} {\St(\e_2)} & \text{App}
  \end{array}
\]

\subsection{Algorithmic Typing}

Jugements of the form:

\[ \algtp \G \St \eu \e \tp \St' \]

\[
  \deduct
  {}
  {\algtp \G \St \unit \unit \unit \St}
  {\text{(Unit)}}
  \spc
  \deduct
  {\G(\var) = \tp}
  {\algtp \G \St \var \var \tp \St}
  {\text{(Var)}}
\]

\[
  \deduct
  {\begin{array}{c}
    \fresh \uv \spc \spc
   \algtp \G {\St_0} {\eu_1} {\e_1} \tarr {\St_1} \\
   \algtp \G {\St_1} {\eu_2} {\e_2} \targ {\St_2} \spc
   \equals {\St_2} {\tarr} {\arr {\targ} \uv} {\St_3}
   \end{array}
  }
  {\algtp \G {\St_0} {\app {\eu_1} {\eu_2}} {\app {\e_1} {\e_2}} {\uv} {\St_3}}
  {\text{(App)}}
\]

\[
  \deduct
  {
    \fresh \uv \spc
    \algtp {\G, \withtp \x {\uv}} {\St_0} \eu \e \tret {\St_1}
  }
  {\algtp \G {\St_0} {\lam \x \eu} {\lamtp \x {\uv} \e} {\arr {\uv} {\tret}} {\St_1}}
  {\text{(Arr)}}
\]

Judgements of the form:

\[ \equals {\St_0} {\tp_1} {\tp_2} {\St_1} \]

\[
  \deduct
  {}
  {\equals \St \tp \tp \St}
  {\text{(Base Eq)}}
  \spc
  \deduct
  {\equals {\St_0} {\tp_1} {\tp_3} {\St_1} \spc
   \equals {\St_1} {\tp_2} {\tp_4} {\St_2}}
  {\equals {\St_0} {\arr {\tp_1} {\tp_2}} {\arr {\tp_3} {\tp_4}} {\St_2}}
  {\text{(Arr Eq)}}
\]

\[
  \deduct
  {\uv \notin \tp \spc \uv \notin \St}
  {\equals \St \uv \tp {\St, \uv = \tp}}
  {\text{(UVar Left Eq)}}
  \spc
  \deduct
  {\uv \notin \tp \spc \uv \notin \St}
  {\equals \St \tp \uv {\St, \uv = \tp}}
  {\text{(UVar Right Eq)}}
\]

\section{Theorems}

\begin{lemma}[Equality]
  If there exists a judgement
  \[ \equals \St {\tp_1} {\tp_2} {\St'},\]
  then under context \(\St'\), the types \(\tp_1\) and \(\tp_2\) are equal.
\end{lemma}

\begin{proof}
  TBD
\end{proof}

\begin{lemma}[Inversion (Decl)]
  If
  \[ \hastp \G \e \tp,\]
  we may deduce what its derivation is unambiguously from the form of
  its expression \(\e\).
\end{lemma}

\begin{proof}
  Follows from the definition.
\end{proof}

\begin{lemma}[Inversion (Algorithmic)]
  If there exists a judgement
  \[ \algtp \G \St \eu \e \tp \St',\]
  we may deduce what its derivation is unambiguously from the form of
  its expression \(\eu\).
\end{lemma}

\begin{proof}
  Follows from the definition.
\end{proof}

\begin{lemma}[Extension 1]
  If
  \[ \equals {\St_0} {\tp_1} {\tp_2} {\St_1}, \]
  then \(\St_0 \subseteq \St_1\).
\end{lemma}

\begin{proof}
  We prove this by induction over the deductions.

  The Base Eq judgement is one of our base cases, and is satisfied by the Refl
  rule of \(\subseteq\). The other two base cases are UVar Left Eq and UVar
  Right Eq, which are satisfied by the Extension rule of \(\subseteq\).
  Informally, we know that \(\St \subseteq \St, \uv = \tp\) because the left is
  a strict prefix of the right and \(\St_0 \subseteq \St_1\) means \(\St_0\) is
  a prefix of \(\St_1\).

  The one recursive case is the Arr Eq judgement, reproduced below.
  \[
  \deduct
  {\equals {\St_0} {\tp_1} {\tp_3} {\St_1} \spc
   \equals {\St_1} {\tp_2} {\tp_4} {\St_2}}
  {\equals {\St_0} {\arr {\tp_1} {\tp_2}} {\arr {\tp_3} {\tp_4}} {\St_2}}
  {\text{(Arr Eq)}}
  \]

  By inversion and the inductive hypothesis, \(\St_0 \subseteq \St_1\) and
  \(\St_1 \subseteq \St_2\). Because \(\subseteq\) is transitive, we may
  conclude that \(\St_0 \subseteq \St_2\), as desired.

  By induction, this property holds for all judgements.
\end{proof}

\begin{lemma}[Extension 2]
  If
  \[ \algtp \G \St \eu \e \tp \St', \]
  then \(\St \subseteq \St'\).
\end{lemma}

\begin{proof}
  We prove this by induction over the deductions.

  Rules Unit and Var constitute base cases, and we observe that the Refl rule of
  \(\subseteq\) satisfies the lemma.

  There are two recursive cases: App and Arr. For Arr, we observe that the
  inductive hypothesis and inversion is sufficient to prove the lemma. For App,
  reproduced below, we will have to do a little work, starting with inversion.

  \[
    \deduct
    {\begin{array}{c}
      \fresh \uv \spc \spc
     \algtp \G {\St_0} {\eu_1} {\e_1} \tarr {\St_1} \\
     \algtp \G {\St_1} {\eu_2} {\e_2} \targ {\St_2} \spc
     \equals {\St_2} {\tarr} {\arr {\targ} \uv} {\St_3}
     \end{array}
    }
    {\algtp \G {\St_0} {\app {\eu_1} {\eu_2}} {\app {\e_1} {\e_2}} {\uv} {\St_3}}
    {\text{(App)}}
  \]

  We may apply the inductive hypothesis to conclude \(\St_0 \subseteq \St_1\)
  and \(\St_1 \subseteq \St_2\). We may apply the Extension 1 lemma to conclude
  \(\St_2 \subseteq \St_3\). By the transitive property, we may conclude \(\St_0
  \subseteq \St_3\), as desired.

  By induction, this property holds for all judgements.
\end{proof}

\begin{lemma}[Substitution]
  If
  \[\hastp \G \e \tp, \]
  we may substitute every occurrence of the UVar \(\uv\) with \(\tp\) and produce a valid
  judgement. Formally, we may derive the following judgement
  \[\left( \hastp \G \e \tp \right)[\uv \mapsto \tp].\]
\end{lemma}

\begin{proof}
  We observe that if \(\uv\) is not present in the judgement, the lemma holds
  trivially. Likewise, if \(\uv\) is in \(\G\) but on an unused typing variable,
  the lemma holds trivially. The rest of this proof will assume that \(\uv\)
  exists somewhere other than the typing context.

  We prove it by induction on judgements.

  Starting with base cases (and looking at their derivations, by inversion),
  DeclUnit holds as it is on a concrete type which may not be substituted. We
  consider now the other base case: DeclVar, reproduced below.

  \[
    \deduct
    {\G(\var) = \uv}
    {\hastp \G \var \uv}
    {\text{(DeclVar)}}
  \]

  Note that we specify that \(\G(\var) = \uv\), as the only meaningful place to
  substitute for \(\uv\) would be if it were a part of the resulting type. In
  this case, if we substitute \(\uv\) for \(\tp\) in \(\G\), it must be true
  that \(\G(\var) = \tp\). From this, we may conclude that
  \[ \hastp {\G[\uv \mapsto \tp]} {\var} {\tp}, \]
  which is a less complex way of stating the desired (below).
  \[ \hastp {\G[\uv \mapsto \tp]} {\var[\uv \mapsto \tp]} {\uv[\uv \mapsto \tp]} \]

  Now for the recursive cases, DeclLam and DeclApp. We'll provide a thorough
  derivation of DeclLam; DeclApp follows in a similar manner.  Suppose we have a
  judgement of the form

  \[\hastp \G {\lamtp {\x} {\tp_1} {\e}} {\arr {\tp_1} {\tp_2}}.\]

  We'd like to know that substituting \(\tp\) for \(\uv\) in this judgement will
  produce another valid judgement. By inversion, we know the following
  derivation exists

  \[
  \deduct
  {\hastp {\G, {\withtp \x {\tp_1}}} \e {\tp_2}}
  {\hastp \G {\lamtp {\x} {\tp_1} {\e}} {\arr {\tp_1} {\tp_2}}}
  {\text{(DeclLam)}}.
  \]

  By the inductive hypothesis, we know that

  \[ \left(\hastp {\G, {\withtp \x {\tp_1}}} \e {\tp_2}\right)[\uv \mapsto \tp]. \]

  If we apply the substitution, we get

  \[ \hastp {\G[\uv \mapsto \tp], {\withtp \x {\tp_1[\uv \mapsto \tp]}}} {\e[\uv
  \mapsto \tp]}{\tp_2[\uv \mapsto \tp]}.\]

  We may apply the DeclLam rule to this judgement to obtain

  \[
    \deduct
    {\hastp {\G[\uv \mapsto \tp], {\withtp \x {\tp_1[\uv \mapsto \tp]}}} {\e[\uv
  \mapsto \tp]}{\tp_2[\uv \mapsto \tp]}}
    {\hastp {\G[\uv \mapsto \tp]} {\lamtp {\x} {\tp_1[\uv \mapsto \tp]} {\e[\uv
    \mapsto \tp]}} {\arr {\tp_1[\uv \mapsto \tp]} {\tp_2[\uv \mapsto \tp]}}}
    {\text{(DeclLam)}}.
  \]

  Skipping a few steps, we see that the resulting judgement is the same as
  \[ \left(\hastp \G {\lamtp {\x} {\tp_1} {\e}} {\arr {\tp_1}
  {\tp_2}}\right)[\uv \mapsto \tp], \]
  as desired.

  DeclApp follows in a similar way, the general step being applying inversion,
  then the inductive hypothesis, then re-deriving the desired judgement.

  By induction, this lemma holds for all valid judgements.

\end{proof}

\begin{lemma}[Weakening]
  If
  \[\hastp {\St(\G)} {\St(\e)} {\St(\tp)}\]
  and know that \(\St \subseteq \St'\), then we may conclude that
  \[\hastp {\St'(\G)} {\St'(\e)} {\St'(\tp)}.\]
\end{lemma}

\begin{proof}
  We prove this lemma by induction on \(\St'\).

  We start with the base case where \(\St = \St'\), which trivially holds.

  Now assume that the judgement holds for all \(\St'\) which are \(n\) elements
  larger than \(\St\). Consider \(\St_{n+1}\), which is \(n+1\) elements larger
  than \(\St\). We may assume that \(\St_{n+1} = \St_n, \uv = \tp\), where we
  know by the inductive hypothesis that weakening holds for \(\St_n\), i.e.

  \[\hastp {\St_n(\G)} {\St_n(\e)} {\St_n(\tp)}.\]

  Our goal is to derive the same for \(\St_{n+1}\). We apply the Substitution
  lemma using the above, \(\uv\), and \(\tp\), to derive

  \[\left(\hastp {\St_n(\G)} {\St_n(\e)} {\St_n(\tp)}\right)[\uv \mapsto \tp].\]

  Since applying a store is the same thing as substitution, we may rewrite the
  above to the identical judgement

  \[\hastp {(\St_n, \uv = \tp)(\G)} {(\St_n, \uv = \tp)(\e)} {(\St_n, \uv = \tp)(\tp)}.\]

  \(\St_n, \uv = \tp = \St_{n+1}\), proving the inductive step.

  By induction, this lemma holds for all \(\St' \supseteq \St\).
\end{proof}

\begin{theorem}[Soundness]
  If
\[ \algtp \G \St \eu \e \tp \St' \]
  then we may derive a declarative judgement
\[ \hastp {\St'(\G)} {\St'(\e)} {\St'(t)}.\]
\end{theorem}

\begin{proof}
  The proof is by induction over the expressions.

  The base cases for the rules Unit and Var are trivial.

  For App, we start with the given judgement, by inversion:

\[
  \deduct
  {\begin{array}{c}
    \fresh \uv \spc \spc
   \algtp \G {\St_0} {\eu_1} {\e_1} \tarr {\St_1} \\
   \algtp \G {\St_1} {\eu_2} {\e_2} \targ {\St_2} \spc
   \equals {\St_2} {\tarr} {\arr {\targ} \uv} {\St'}
   \end{array}
  }
  {\algtp \G {\St_0} {\app {\eu_1} {\eu_2}} {\app {\e_1} {\e_2}} {\uv} {\St'}}
  {\text{(App)}}
\]

  By induction, we know that there are declarative judgements
  \[ \hastp {\St_1(\G)} {\St_1(\e_1)} {\St_1(\tarr)} \]
  and
  \[ \hastp {\St_2(\G)} {\St_2(\e_2)} {\St_2(\targ)}.\]

  We may apply the weakening lemma to conclude 

  \[ \hastp {\St'(\G)} {\St'(\e_1)} {\St'(\tarr)} \]
  and
  \[ \hastp {\St'(\G)} {\St'(\e_2)} {\St'(\targ)}.\]

  By the equality lemma, we know that \(\tarr = \arr \targ \uv\), so we may
  substitute for it:

  \[ \hastp {\St'(\G)} {\St'(\e_1)} {\St'(\arr \targ \uv)}.\]

  By property of substitution, we get

  \[ \hastp {\St'(\G)} {\St'(\e_1)} {\arr {\St'(\targ)} {\St'(\uv)}}.\]

  Thus we may construct the declarative judgement
  
  \[
    \deduct
    {\hastp {\St'(\G)} {\St'(\e_1)} {\arr {\St'(\targ)} {\St'(\uv)}} \spc
     \hastp {\St'(\G)} {\St'(\e_2)} {\St'(\targ)}
    }
    {\hastp {\St'(\G)} {\app {\St'(\e_1)} {\St'(\e_2)}} {\St'(\uv)}}
    {}
  \]

  By property of substitution, we know that \(\app {\St'(\e_1)} {\St'(\e_2)}\) is
  the same as \(\St'(\app {\e_1} {\e_2})\), which gives the desired.

  For Arr, we start with the given judgement, again by inversion:

\[
  \deduct
  {
    \algtp {\G, \withtp \x {\uv}} {\St_0} \eu \e \tret {\St'}
  }
  {\algtp \G {\St_0} {\lam \x \eu} {\lamtp \x {\uv} \e} {\arr {\uv} {\tret}}
  {\St'}}
  {\text{(Arr)}}
\]

  By induction, we can conclude that there exists a derivation for the declarative judgement

  \[ \hastp {\St'(\G), \withtp \x {\St'(\uv)}} {\St'(\e)} {\St'(\tret)}.\]

  From this, we may derive the declarative judgement

  \[
    \deduct
    {\hastp {\St'(\G), \withtp \x {\St'(\uv)}} {\St'(\e)} {\St'(\tret)}}
    {\hastp {\St'(\G)} {\lam \x {\St'(\e)}} {\arr {\St'(\uv)} {\St'(\tret)}}}
  \]
  
  By the property of substitution, we know that \({\arr {\St'(\uv)}
  {\St'(\tret})}\) is the same as \(\St'(\arr {\uv} {\tret})\), which gives
  the desired.

\end{proof}

\section{To Do}

- Replace subseteq with square subseteq (and supseteq)
- Replace occurrences of substitution \([\uv \mapsto \tp]\) with a singleton
type store of \(\nil, \uv = \tp\). Have substitution lemma be \(\hastp {\St(\G)}
{\St(\e)} {\St{\tp}} \implies \hastp {(\St, \uv = \tp')(\G)} {(\St, \uv =
\tp')(\e)} {(\St, \uv = \tp')(\tp)}\).
- Clean up DeclVar case in Substitution.
- Make substitution apply stores from beginning to end. (S,u1 = t')(t) :=
S(t)[u1 |-> t'].
- See about writing up the judgements as witnesses in LH.

two options for proving infer works
- Make 'infer' take \(\Gamma\), \(\St\), and \(\eu\) as input and return \(\e\),
\(\tp\), \(\St'\) and a witness that of the declarative judgement (from
soundness).
- Take the same inputs and only return the first three outputs. Write a proof
that the function produces the delcarative judgement itself.

\end{document}
